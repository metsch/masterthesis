1920 wurde eine historische Aufzeichnung von Vladimir Lenin und Leo Trotzki verändert und dadurch, mit Farbe und Pinsel die Geschichte gefälscht. Langfristig aufbewahrte Daten in digitalen Archiven bieten böswilligen Akteuren die Möglichkeit, Daten zu manipulieren. Datenmanipulationen in digitalen Archiven werden in der Regel durch den Vergleich kryptographischer Hash-Werte in unterschiedlichen Zeitintervallen erkannt, sogenannten Fixitätsinformationen. Während das Generieren von Fixitätsinformationen (z. B. MD5, SHA256) relativ einfach ist, ist das Speichern schwieriger, wenn man bedenkt, dass jeder, der sie ändern kann, auch die zugrunde liegenden Daten ändern kann.
Jüngste Arbeiten haben gezeigt, dass die Ethereum-Blockchain geeignet ist, Metadaten unveränderlich zu speichern, was einen guten Kandidaten für die Speicherung von Fixitätsinformationen darstellt. Aber die Kosten einer individuellen Teststrategie, bei welcher die Fixitätsinformationen für jedes Objekt einzeln persistiert werden, ist zu hoch. Eine Strategie, um den Kosten entgegenzuwirken, besteht darin, die Anzahl kostspieliger Transaktionen auf der Ethereum-Blockchain zu minimieren. Dies kann durch den Einsatz von Pool-Testing erfolgen, was erstmals 1943 präsentiert wurde und heutzutage zum Untersuchen von Populationen auf COVID-19 verwendet werden, indem einzelne Proben zu einem Pool kombiniert werden. Dieses Konzept kann auch mit Hash-Listen implementiert werden, bei denen die kryptografischen Hashes digitaler Objekte in einer Hash-Liste kombiniert werden und nur der Root-Hash auf der Blockchain persistiert werden muss. Dadurch wird die Anzahl kostspieliger Transaktionen auf der Blockchain minimiert, während die Effizienz erhalten bleibt. Durch die Verwendung von Pool-Testing können die Kosten zur Sicherstellung der Integrität des OpenPreserve format-corpus Datensatzes auf der Ethereum-Blockchain um den Faktor 3,0 reduziert und die Effizienz um den Faktor 1,28 gesteigert werden.
Diese Arbeit endet mit der Erfassung des genauen Rechenaufwands und der Kosten für den Betrieb eines Fixitätsspeichers auf der Ethereum-Blockchain und einer Methode, um die Kosten dafür zu senken.
