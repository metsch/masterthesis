Diese Masterarbeit gibt einen Überblick über digitale Archive und das Konzept von Vertrauen, Authentizität und Integrität von Daten in Archiven und eine Methode, welche die Blockchain-Technologie nutzt, um diese Konzepte zu stärken. In einer modernen Zeit, in der das Fälschen und Manipulieren von Daten einfacher denn je ist, ist es umso wichtiger, das Vertrauen in öffentliche Einrichtungen zu stärken, in denen Kulturgüter aufbewahrt werden.
In meiner Masterarbeit stelle ich einen dezentralen Fixitätsspeicher auf der Ethereum-Blockchain vor, Fixität ist ein Attribut eines digitalen Objekts, mit welchem validiert werden kann, dass ein Objekt über einen gewissen Zeitraum nicht manipuliert wurde. Die Ethereum-Blockchain eignet sich so gut als Fixitätsspeicher, weil sie einen neutralen Dritten darstellt, der keinen Vorteil aus der Manipulation der in seiner Obhut befindlichen Daten hat; mit der Fähigkeit und den Mitteln hat, dies zu beweisen und somit eine eine Verschiebung vom Vertrauen in Behörden zum Vertrauen in die Technologie bewirkt. Duranti (1998, 36) sagte, dass in der Antike „Authentizität kein intrinsisches Merkmal von Dokumenten war, sondern die Tatsache, wo die Dokumente aufbewahrt wurden, in einem Archiv, einem Tempel oder einer respektierten öffentlichen Institution“.