In dieser Arbeit geht es darum die Datenintegrität eines Digitales Archives, in welchem Kulturgüter wie Bilder oder Texte archiviert sind, mit Hilfe von blockchain Technologie zu sichern. Ich verschaffe einen Überblick über digitale Archive und das Konzept von Vertrauen, Authentizität und Integrität von Daten in Archiven und eine Methode, um diese Konzepte zu stärken. In der modernen Zeit, in der das Fälschen und Manipulieren von Daten einfacher denn je ist, ist es umso wichtiger, das Vertrauen in öffentliche Einrichtungen wie Archive zu stärken.
Um die Datenintegrität zu gewährleisten, werden üblicherweise Kryptografische Verfahren angewandt, um sogenannte Fixitäts Informationen zu erstellen. Fixität ist ein Attribut eines digitalen Objekts, mit welchem validiert werden kann, dass ein Objekt über einen gewissen Zeitraum nicht manipuliert wurde. Auch wenn das Erstellen von Fixitäts Informationen (z.B. MD5, SHA256) relativ einfach ist, ist das Speichern von dessen umso schwerer, wenn Sie bedenken, dass jeder der diese ändern kann auch die zu Grunde liegenden Daten ändern kann.
Als Speichermedium für die Fixität stelle Ich in dieser Arbeit die Ethereum blockchain vor. Es wurde bereits gezeigt, dass die Ethereum blockchain geeignet ist, um Metadaten unveränderbar zu persistieren, jedoch sind die Kosten für eine Individuelle Strategie, in welcher die Metadaten jedes Objektes einzeln in der Blockchain persistiert werden, immens.
Um den Kosten entgegenzuwirken, werde Ich eine Pool-Testing-Strategie anwenden, in welcher mehrere digitale Objekte in einem Pool zusammengefasst werden. Somit wird die Anzahl an kostenpflichtigen Transaktionen auf der blockchain minimiert, wobei trotzdem die Effizienz gewahrt wird. Die Idee zu diesem Vorgehen stammt aus der derzeitigen Pandemie, in der die Testkapazitäten optimal genutzt werden müssen.
In der Evaluation berücksichtige Ich die verursachten Kosten und die Effizienz einer Pooling-Strategie im Vergleich zur individuellen Strategie. Ich zeige, anhand des OpenPreserve Format Korpusses, dass die Anzahl an benötigten Fixitäts Tests um das 2.25-Fache und die Kosten um das 5-Fache durch eine Pooling-Strategie reduziert werden können. Ich zeige zudem eine Kontext sensitive Pooling-Strategie, welche die Volatilität einzelner Dateiformate berücksichtigt und dadurch angepasste Poolgrößen anwendet. Die Kontext sensitive Methode ist ausbalancierter mit einer Testreduzierung um das 3.28-Fache und einer Kostenreduzierung um das 3.48-Fache.
Der Beitrag dieser Arbeit zu diesem Forschungsfeld ist die exakte Kostenberechnung eines dezentralen Fixitätsspeichers auf der Ethereum blockchain und zusätzlich eine Methode, um die Kosten eines solchen zu reduzieren.

