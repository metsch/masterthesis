This work is about securing the data integrity of a digital archive, in which cultural assets such as images or text are stored, with the help of blockchain technology. I provide an overview of digital archives and the concepts of trust, authenticity, and integrity of data in archives and a method to strengthen these concepts. In a modern age, where falsifying and manipulating data is easier than ever, it is even more important to strengthen trust in public institutions.
To ensure data integrity, cryptographic methods are usually used to create so-called fixity information. Fixity is an attribute of a digital object that can be used to validate that an object has not been tampered with for a certain time interval. While generating fixity information (e.g., MD5, SHA256) is relatively easy, storing it is more difficult when you consider that anyone who can change it can also change the underlying data.
In this work, I present the Ethereum blockchain as a storage medium for the fixity information. It has already been shown that the Ethereum blockchain is suitable for persisting metadata in an unchangeable manner, but the costs for an individual strategy in which the metadata of each object is persisted individually in the blockchain are immense.
To counteract the costs, I utilize a pool testing strategy in which several digital objects are pooled. This minimizes the number of chargeable transactions on the blockchain while still maintaining efficiency. The idea for this approach stems from the ongoing pandemic, in which the test capacities must be used optimally.
In the evaluation, I consider the costs incurred and the efficiency of a pooling strategy compared to an individual strategy. I show, using the OpenPreserve format corpus, that the number of required fixity tests can be reduced by a factor of 2.25 and the cost by a factor of 5 by utilizing a pooling strategy. I also show a context-sensitive pooling strategy that considers the volatility of individual file formats and thus applies adjusted pool sizes. The context-sensitive method is more balanced with a test reduction of 3.28x and a cost reduction of 3.48x.
The contribution of this work to this research field is the exact cost computation of a decentralized fixity storage on the Ethereum blockchain and additionally a method to reduce the costs of such.