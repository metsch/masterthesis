\section{Setup}
The proof of concept is written in Python in form of a Jupyter Notebook. It can be found on GitHub \url{https://github.com/metsch/masterthesis/blob/main/src/py/poc_ropsten.ipynb}. With Python's Web3.py I get the contract from the blockchain and extract its functions setPoolHash and getPoolHash. For the live experiment on the Ropsten test network I have initialized 100 digital objects, in form of SHA256 values, with a prevalence rate of 0.1 (or 10\%). The less amount of objects in the live experiment is due to the fact, that the Ropsten ETH token is hard to get, since the most relevant publicly available faucets are drained out and the ones available have a rate of only 0.1 ETH per day, as I explained in Section \ref{sec:test-nets}. I computed the optimal pool size k for N=100 and p=0.1 with Equation \ref{eq:poolsize} and created $\lceil N/k \rceil$ pools. Each pool is assigned an ID and has an array of objects, from which a hash list is formed, and the resulting root hash is stored. 
Before uploading the pools on to the blockchain, I estimated the gas cost for a single transaction in order to set the gas limit to avoid overpaid transactions. I utilized Equation \ref{eq:tx-data} and added 20\% in order guarantee that the transaction is not underpaid, since the result of Equation \ref{eq:tx-data} is the exact gas amount and may be reverted.
At the time of the experiment, the gas price was 86.11 gwei which is 0,0000000896 ETH, resulting in \$15.18 for a setPoolHash transaction calculated with Equation \ref{eq:tx-cost} with an ETH price of \$4000. Admitted, this gas price is really high, usually the gas price is about 40 gwei. 
For each pool, 25 in total, I uploaded each root hash on to the blockchain and waited for the transaction to finish and run some tests to see if the transactions were successful.
After uploading the root hashes, I artificially corrupted the objects in the archive with a Bernoulli trial where an object gets corrupted if a random number between 0 and 1 is below p=0.1. At last, I repaired the archive by checking the local pools with the ones on the blockchain, and if the pool hashes did match the pool is seen as uncorrupted. Where the pools with non-matching hashes got replaced by copies of the objects.
To sum it up, the cost for the live experiment was for 25 writing transactions 0.11204969029 ETH (\$448) and 8 data-scrubbing operations resulting in 33 total operations. Whereas, with an individual testing strategy the amount of writing transactions would have been 100 and estimated 10 data-scrubbing operations. The experiment showed no weakness in throughput, where 25 transactions were uploaded in an instant and mined in less than 2 minutes