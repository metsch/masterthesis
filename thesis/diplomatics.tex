\section{About Diplomatics}
Diplomatics is a science founded in France in the seventeenth century by Benedictine monk Dom Jean Mabillon in his dissertation De Re Diplomatica Libri VI (1681) to determine the provenance and authenticity of evidence attesting to patrimonial rights. It was then utilized by attorneys to settle disputes, historians to interpret documents, and editors to publish medieval deeds and charters, and it evolved into a legal, historical, and philological specialty.
Classic diplomatics and modern/digital diplomatics should be distinguished since these two areas of the field do not represent a natural evolution of the latter from the former, but rather operate in parallel and focus on separate topics. Modern diplomatics encompasses all documents produced in the course of any kind of business, and is defined as "the discipline that studies the genesis, forms, and transmission" of records, as well as "their relationship with the facts represented in them and with their creator, in order to identify, evaluate, and communicate their true nature."

Both classic and modern diplomatics are concerned with determining the trustworthiness of records; however, the former does so retrospectively, examining records from several centuries ago, whereas the latter is concerned not only with establishing the trustworthiness of historical information but also with ensuring the trustworthiness of records yet to be created \cite[10]{kirschenbaum2010digital}. 

A digital record is a digital component, or group of digital components, that is saved, treated, and managed as a record, or, more specifically, a record whose content and form are encoded using discrete numeric values, such as the binary values 0 and 1, rather than a continuous spectrum of values, as defined by modern diplomatics. A digital record differs from analogue and electronic records in that it is digital. The representation of an item or physical process using continuously varying electronic signals or mechanical patterns is referred to as analogue by InterPARES\footnote{InterPARES 2, Terminology Database, \url{http://www.interpares.org/ip2/ip2_terminology_db.cfm} (accessed on 25 April 2022)}.
An analogue depiction of an object or physical process, in contrast to a digitally recorded representation, closely mimics the original. Any analogue or digital record conveyed by an electrical conductor and requiring the use of electronic equipment to be comprehensible by a person is defined as an electronic record by InterPARES. Using the classic archive idea, InterPARES defines a record as a document created or received as an instrument or by-product of a practical activity and set aside for action or reference \cite[44]{duranti2009digital}.

The integrity of a record is determined not only by its appearance – which can be deceiving in the case of good forgeries – but also by the circumstances of its maintenance and preservation: until proof to the contrary, an unbroken chain of responsible and legitimate custody is considered an insurance of integrity, and integrity metadata are required to attest to that. The duty for a record's authenticity changes from the creator's trusted record keeper, who must guarantee it for the duration of the record's custody, to the trusted custodian, who must guarantee it for the duration of the record's existence \cite[53]{duranti2009digital}. In this thesis the Ethereum network acts as a trusted custodian which guarantees that no one is able to tamper with the fixity information of digital records.

There are no originals in the diplomatic sense in the digital environment since there are no records that are the first instance of an item. When a digital record is closed for the first time, the original is destroyed and every time it opens a copy is created. However, it can be stated that each digital record is a copy in the latest version utilized by the creator, and that any version retained by the preserver is a genuine copy of the creators record.

\section{Trustworthiness}
Whether it is a stack of paper or a digital file, trustworthiness is a concept and an obligation that lasts the life of the document. As files go through the stages of the preservation process, from initial capture and metadata extraction to longer-term strategies like migration and rights management, the needs of born-digital objects change. Born-digital objects follow a similar path from creator to intermediary, such as a dealer or other (human or technology) agent, to archival repository staff, and eventually to storage and, possibly, ingest into a digital repository. The phases of that journey make up a digital objects chain of custody, and each one has a significant impact on the trustworthiness of the born-digital elements in a given accession.\newline The fundamental parts of creating and maintaining trust are the provenance of both analog and digital resources, as well as documentation concerning their storage environment, what has been done to them, and by whom. In the collection and preservation of born-digital resources, the trustworthiness of an institution, a custodian, or a document is critical \cite[27]{kirschenbaum2010digital}. 

Modern diplomatics concerns itself with four aspects of trustworthiness: reliability, authenticity, accuracy and authentication \cite[10]{kirschenbaum2010digital}.

\subsection{Reliability}
The content of a record's trustworthiness as a statement of fact. It is evaluated on the basis of 1) the completeness of the record, that is, the presence of all formal elements required by the legal-administrative system for that specific record to be capable of achieving the purposes for which it was created; and 2) the controls exercised on the process of creating the record, among which are those exercised on the author of the record, who must be the person competent, that is, having the authority and capacity to create the record.
The creator and trustworthy record keeper, that is, the person or organization who made or received the record and kept it with its other records, are solely responsible for its reliability \cite[52]{duranti2009digital}.

Archivist' expectations reliable sources have evolved over time, from the belief that librarians and archivists would provide researchers with verifiable evidence, to more modern understandings that reject the ideal of the reliable source and regard all texts as potentially deceptive and richly ambiguous. Data-stewards should be able to offer researchers with documentation about the provenance and acquisition of the data in their care using the methods of operation and processes created by the community and professionals, such as a provenance chain or the fixity information created on ingest of the object in question \cite[32]{kirschenbaum2010digital}.

Open source software is favorable in terms of reliability because community members can look into the code and verify that no unwanted actions will be executed. From a technical perspective, the blockchain suits as a source of reliability since it is well accepted by the public and every transaction ever made is publicly viewable and therefore fulfills the first point of reliability: completeness of the record. Also, the source code of the Ethereum network and each update (fork) gets peer reviewed and is available on GitHub\footnote{\url{https://github.com/ethereum/go-ethereum}}.


\subsection{Authenticity}
Authenticity refers to a record's ability to be trusted as a record, and is defined as the fact that a record has not been tampered with or corrupted, either accidentally or with malicious intend. An authentic record keeps the same identity it had when it was created and can be presumed or demonstrated to have kept its integrity across time. The identity of a record is made up of all the features that set it apart from other records, and it is determined by the formal components on the record's face and/or its attributes, as stated in a register entry or as metadata \cite[52]{duranti2009digital}.
With the fixity information stored on the blockchain, it can be proven that a record in the archive has not been tampered with. The simple comparison of the hash values generated in different points in time can prove the authenticity of a record.

\subsection{Accuracy}
Because record correctness was subsumed under both trustworthiness and authenticity as a notion, it was never a consideration in general diplomatics. Accuracy is defined as the truthfulness, exactness, or completeness of the data (i.e., the smallest, meaningful, indivisible pieces of information) within a record. Because of the ease with which data can be damaged during transmission over space (between humans and/or systems) and time in the digital environment, accuracy must be considered and assessed as a separate quality of a record (when digital systems are upgraded or records are migrated to a new system). As a result, the responsibility for accuracy shifts over time from the creator's trusted record-keeper to the trusted custodian \cite[52]{duranti2009digital}.
Whether it is a bit-rot or any other unintended error during the preservation process, a cryptographic hash value will be different if any bit of the object in the archive has changed and therefore the system presented in this thesis fulfills the requirements of accuracy.

\subsection{Authentication}
Authentication is defined as a statement or an element, such as a seal, a stamp, or a symbol, attached to a record after it has been completed by a competent employee. Authentication simply assures that a record is authentic at one precise point in time, when the declaration is made or the authenticating element or entity is affixed, whereas authenticity is a quality of the record that accompanies it for as long as it remains as is. A digital signature is frequently used in the digital environment to give ultimate authenticity. Because it is tied to a full record, the digital signature serves as a seal, allowing verification of the record's origin and integrity, as well as making the record unassailable and incontestable by performing a non-repudiation function. \cite[53]{duranti2009digital}.
Authentication in this process is not intended, since I do not use any form of digital signature to prove that a record is indeed authentic on ingest. In this thesis, I assume that every object is authentic and reviewed by a competent employee.

\section{Summary}
The concept of trust in a public authority is presented in this chapter and acts as a foundation for designing the fixity storage. The concept of authenticity is implemented through the usage of SHA256 values in this thesis, where the collision resistance of the SHA256 algorithm guarantees that no object in the archive can be tampered with. Accuracy is also guaranteed through the usage of a cryptographic hash function, since the hash is completely different if a bit level error happens over the course of the preservation.
Reliability is enforced by the nature of the blockchain, where every state changing action is documented immutable as long as the blockchain exists, as later presented in Chapter \ref{ch:ethereum}. Each of the core concepts of trust is needed in the modern digital world, where it is easier than ever to manipulate data and opinions.