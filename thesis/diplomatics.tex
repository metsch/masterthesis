\section{About Diplomatics}
Diplomatics is a science founded in France in the seventeenth century by Benedictine monk Dom Jean Mabillon in his dissertation De Re Diplomatica Libri VI (1681) to determine the provenance and authenticity of evidence attesting to patrimonial rights. It was then utilized by attorneys to settle disputes, historians to interpret documents, and editors to publish medieval deeds and charters, and it evolved into a legal, historical, and philological specialty.
Classic diplomatics and modern/digital diplomatics should be distinguished since these two areas of the field do not represent a natural evolution of the latter from the former, but rather operate in parallel and focus on separate topics. Modern diplomatics encompasses all documents produced in the course of any kind of business, and is defined as "the discipline that studies the genesis, forms, and transmission" of records, as well as "their relationship with the facts represented in them and with their creator, in order to identify, evaluate, and communicate their true nature."

Both classic and modern diplomatics are concerned with determining the trustworthiness of records; however, the former does so retrospectively, examining records from several centuries ago, whereas the latter is concerned not only with establishing the trustworthiness of historical information but also with ensuring the trustworthiness of records yet to be created \cite[10]{kirschenbaum2010digital}. 

A digital record is a digital component, or group of digital components, that is saved, treated, and managed as a record, or, more specifically, a record whose content and form are encoded using discrete numeric values, such as the binary values 0 and 1, rather than a continuous spectrum of values, as defined by modern diplomatics. A digital record differs from analogue and electronic records in that it is digital. The representation of an item or physical process using continuously varying electronic signals or mechanical patterns is referred to as analogue by InterPARES.
An analogue depiction of an object or physical process, in contrast to a digitally recorded representation, closely mimics the original. Any analogue or digital record conveyed by an electrical conductor and requiring the use of electronic equipment to be comprehensible by a person is defined as an electronic record by InterPARES. Using the classic archive idea, InterPARES defines a record as a document created or received as an instrument or by-product of a practical activity and set aside for action or reference \cite[52]{duranti2009digital}.

The integrity of a record is determined not only by its appearance – which can be deceiving in the case of good forgeries – but also by the circumstances of its maintenance and preservation: until proof to the contrary, an unbroken chain of responsible and legitimate custody is considered an insurance of integrity, and integrity metadata are required to attest to that. The duty for a record's authenticity changes from the creator's trusted record keeper, who must guarantee it for the duration of the record's custody, to the trusted custodian, who must guarantee it for the duration of the record's existence \cite[53]{duranti2009digital}. In this thesis the Ethereum network acts as a trusted custodian which guarantees that no one is able to tamper with the fixity information of digital records.

\section{Trustworthiness}
Whether it's a stack of paper or a digital file, trustworthiness is a concept and an obligation that lasts the life of the document. As files go through the stages of the preservation process, from initial capture and metadata extraction to longer-term strategies like migration and rights management, the needs of born-digital objects change. Born-digital fonds follow a similar path from creator to intermediary, such as a dealer or other (human or technology) agent, to archival repository staff, and eventually to storage and, possibly, ingest into a digital repository. The phases of that journey make up a digital object's chain of custody, and each one has a significant impact on the trustworthiness of the born-digital elements in a given accession.\newline The fundamental parts of creating and maintaining trust are the provenance of both analog and digital resources, as well as documentation concerning their storage environment, what has been done to them, and by whom. In the collection and preservation of born-digital resources, the trustworthiness of an institution, a custodian, or a document is critical \cite[27]{kirschenbaum2010digital}. 

Modern diplomatics concerns itself with five aspects of trustworthiness: reliability, authenticity, accuracy, integrity and authentication \cite[10]{kirschenbaum2010digital}.

\subsection{Reliability}
The content of a record's trustworthiness as a statement of fact. It is evaluated on the basis of 1) the completeness of the record, that is, the presence of all formal elements required by the legal-administrative system for that specific record to be capable of achieving the purposes for which it was created; and 2) the controls exercised on the process of creating the record, among which are those exercised on the author of the record, who must be the person competent, that is, having the authority and capacity to create the record.
The creator and trustworthy record keeper, that is, the person or organization who made or received the record and kept it with its other records, are solely responsible for its reliability \cite[52]{duranti2009digital}.

Open source software is asked because community members can look into the code and verify that no unwanted actions will be executed. The blockchain, in regard to reliability, is well accepted by the public and every transaction ever made is publicly viewable. Also, the source code of the Ethereum network and each update (fork) gets peer reviewed and is available on GitHub\footnote{\url{https://github.com/ethereum/go-ethereum}}.


\subsection{Authenticity}
Authenticity refers to a record's ability to be trusted as a record, and is defined as the fact that a record has not been tampered with or corrupted, either accidentally or with malicious intend. An authentic record keeps the same identity it had when it was created and can be presumed or demonstrated to have kept its integrity across time. The identity of a record is made up of all the features that set it apart from other records, and it is determined by the formal components on the record's face and/or its attributes, as stated in a register entry or as metadata \cite[52]{duranti2009digital}.

Archivist' expectations of source reliability have evolved over time, from the belief that librarians and archivists would provide researchers with verifiable evidence, to more modern understandings that reject the ideal of the reliable source and regard all texts as potentially deceptive and richly ambiguous. Data-stewards should be able to offer researchers with documentation about the provenance and acquisition of the data in their care using the methods of operation and processes created by the community and professionals, such as a provenance chain or the fixity information created on ingest of the object in question \cite[32]{kirschenbaum2010digital}.

\subsection{Accuracy}
Because record correctness was subsumed under both trustworthiness and authenticity as a notion, it was never a consideration in general diplomatics. Accuracy is defined as the truthfulness, exactness, accuracy, or completeness of the data (i.e., the smallest, meaningful, indivisible pieces of information) within a record. Because of the ease with which data can be damaged during transmission over space (between humans and/or systems) and time in the digital environment, accuracy must be considered and assessed as a separate quality of a record (when digital systems are upgraded or records are migrated to a new system). As a result, the responsibility for accuracy shifts over time from the creator's trusted record-keeper to the trusted custodian \cite[14]{kirschenbaum2010digital}.

\subsection{Integrity}
There are no originals in the diplomatic sense in the digital environment, that is, there are no records that are the first instance of each item under consideration, in addition to being complete and capable of reaching the purposes for which they were generated, because when we close a digital record for the first time, we destroy the original and every time we open it, we create a copy. However, we can state that each digital record is a copy in the form of original in the latest version utilized by the creator in the usual and ordinary course of business, and that any version retained by the preserver is a genuine copy of the creator's record.
If their identity is intact and their integrity can be presumed or demonstrated, they are both authoritative and authentic. When extracting digital evidence, digital forensics must first avoid modifying the data and must be repeatable in order to be considered reliable. The accurate documenting of each and every operation performed on the evidence supports repeatability, which is one of the key axioms of digital forensics practice \cite[58]{duranti2009digital}. 

Such a chain of alteration is possible on the Ethereum network, since each update on an object hash is natively documented on the blockchain.
Duplication integrity is ensured when given a data set, the process of creating a duplicate of the data does not modify the data (either intentionally or accidentally) and the duplicate is an exact bit copy of the original data set. It is possible to preserve data integrity over the duplicate, with respect to the original, by using a trusted third party. At the time the image is created, a copy of the hash can be given to a trusted third party to hold in escrow. Now changes to the duplicate can be detected even if the original is modified.\newline
Experts in digital forensics have also linked duplication integrity to time, and have investigated using time stamps to achieve this. A distinction between the integrity of a record as a whole and the integrity of its duplicate may be useful in resolving the conflict between diplomats' and information technology experts' views on integrity, which tend to support the need for the extreme authentication provided by a digital signature. Indeed, by accepting the digital forensics specialists' proposed link between integrity and time, and defining record integrity differently in each step of the record life cycle and/or custodial history, one could further enrich the concept of integrity \cite[60]{duranti2009digital}.

Both diplomatics and forensics were created as methods for examining existing material evidence, determining its transmission status, legitimacy, and ability to offer confirmation of the circumstances at hand \cite[64]{duranti2009digital}.

\subsection{Authentication}
Authentication is defined as a statement or an element, such as a seal, a stamp, or a symbol, attached to a record after it has been completed by a competent officer. Authentication simply assures that a record is authentic at one precise point in time, when the declaration is made or the authenticating element or entity is affixed, whereas authenticity is a quality of the record that accompanies it for as long as it remains as is. A digital signature is frequently used in the digital environment to give ultimate authenticity. Because it is tied to a full record, the digital signature serves as a seal, allowing verification of the record's origin and integrity, as well as making the record unassailable and incontestable by performing a non-repudiation function. \cite[53]{duranti2009digital}.

With regard to authentication, the method presented in this thesis guarantees that the fixity information of an object is never altered unnoticed by an unauthorized actor, since every transaction can be viewed attached with a timestamp on the blockchain.

\section{Wrap up}
The concept of trust in a public authority is presented in this Chapter and acts as a foundation for designing the fixity storage. Concepts, such as integrity and accuracy are implemented in the form of SHA256 hashes, with which the integrity of an object can be validated at bit level. Reliability is enforced by the nature of the blockchain, where every state changing action is documented immutable as long as the blockchain exists, as later presented in Chapter \ref{ch:ethereum}. Each of the core concepts of trust is needed in the modern digital world, where it is easier than ever to manipulate data and opinions.