Diplomatics is a science that was developed in iplomatics is a science that was developed in France in the seventeenth century by the Benedictine
monk Dom Jean Mabillon in a treatise Diplomatics is a science that was developed in France in the seventeenth century by the Benedictine monk Dom Jean Mabillon in a treatise entitled De Re Diplomatica Libri VI (1681) for the purpose of ascertaining the provenance and authenticity of records
that attested to patrimonial rights. It later grewinto a legal, historical, and philological discipline as itcame to be used by lawyers to resolve disputes, by historians to interpret records, and by editors to publish medieval deeds and charters. It is useful to distinguish classic diplomatics from modern/digital diplomatics, because these two branches of the discipline do not represent a natural evolution of the latter from the former, but exist in parallel and focus on different objects of study. Modern diplomatics has a broader scope; it is concerned with all documents that are created in the course of affairs of any kind, and is define as “the discipline which studies the genesis, forms, and transmission” of records, and “their relationship with the facts represented in them and with their creator, in order to identify, evaluate, and communicate their true nature”. The primary focus of both classic and modern diplomatics is to assess the trustworthiness of records; however, the former establishes it retrospectively, looking at records issued several centuries ago, while the latter is concerned not only with establishing the trustworthiness of existing records but also with ensuring the trustworthiness of records that have yet to be created. \cite[10]{kirschenbaum2010digital}. 
In my thesis, digital diplomatics is the focus.
Digital diplomatics defines a digital record as a digital component, or group of digital components, that is saved, and treated and managed as a record, or, more specifically, a record whose content and form are encoded using discrete numeric values, such as the binary values 0 and 1, rather than a continuous spectrum of values. A digital record is distinguished from an analogue record and an electronic record. InterPARES considers analogue the representation of an object or physical process through the use of continuously variable electronic signals or mechanical patterns. In contrast to a digitally encoded representa tion of an object or physical process, an analogue representation resembles the original. InterPARES defines an electronic record as any analogue or digital record that is carried by an electrical conductor and requires the use of electronic equipment to be intelligible by a person. InterPARES defines a record, using the traditional archival concept, as a document made or received in the course of a practical activity as an instrument or a by-prod uct of such activity, and set aside for action or reference \cite[52]{duranti2009digital}.
The integrity of a record is inferred not only from its appearance – which might be deceiving in the case of good forgeries – but also from the circumstances of its maintenance and preservation: an unbroken chain of responsible and legitimate custody is considered an insurance of integrity until proof to the contrary, and integrity metadata are required to attest to that. The authenticity of a record is a movable responsibility, as it shifts from the creators trusted recordkeeper, who needs to guarantee it for as long as the record is in its custody, to the trusted custodian, who guarantees it for as long as the record exists. \cite[53]{duranti2009digital}. In my thesis the Ethereum network acts as a trusted custodian which guarantees that no one is able to tamper with the cryptographic hash values of digital records.

\section{Trustworthiness}
Trustworthiness is a concept and an obligation that spans the life of a document, whether it is a sheaf of paper or a WordPerfect file. The needs of born-digital objects shift as files move through the stages of the preservation process, from initial capture and metadata extraction to longer-term strategies such as migration and rights management. Born-digital fonds are similarly mobile as they pass from the creator, to an intermediary such as a dealer or other agent (human or technological), to staff at an archival repository, and, finally, to storage and, perhaps, ingest into a digital repository. The stages of that journey constitute the chain of custody for a digital object, and each stage has important implications for the trustworthiness of the born-digital materials in a given accession. Authentic source may be deceptive or unreliable, and although reliability is an important component of trustworthiness, the veracity of a documents content is often not the concern of archivists working with cultural heritage materials. Rather, the provenance of both analog and digital materials, as well as documentation about their storage environment, what has been done to them, and by whom, are the key aspects of establishing and maintaining trust. Trustworthiness— of an institution, a custodian, or a document—plays an important role in the acquisition and maintenance of born-digital materials. How best to determine and document that quality in a digital environment and with regard to the stewardship of born-digital materials is a question that remains under consideration \cite[27]{kirschenbaum2010digital}. 
Digital diplomatics concerns itself with five aspects of trustworthiness: reliability, authenticity, accuracy, integrity and authentication \cite[10]{kirschenbaum2010digital}.

\subsection{Reliability}
Reliability is the trustworthiness of a record as a statement of fact, as to content. It is assessed on the basis of 1) the completeness of the record, that is, the presence of all the formal elements required by the juridical-administrative system for that specific record to be capable of achieving the purposes for which it was generated; and 2) the controls exercised on the process of creation of the record, among which are included those exercised on the author of the record, who must be the person competent, that is, having the authority and the capacity, to issue it. The reliability of a record is the exclusive responsibility of its creator and the trusted recordkeeper, that is, of the person or organization that made or received it and maintained it with its other records \cite[52]{duranti2009digital}.
The concept of reliability, used in reference to the source of the records, is defined in digital forensics in a way that points to a reliable software, measured by principles similar to those the courts use to determine evidentiary reliability, that is, empirical testing, subjection to peer review and publication, determination of error rate, and general acceptance within the relevant community. Also these principles point to open source software because the processes of records creation and maintenance can be authenticated with evidence either by describing a process or system used to produce a result, or by showing that the process or system produces an accurate result \cite[59]{duranti2009digital}.

\subsection{Authenticity}
Authenticity is the trustworthiness of a record as a record, and is defined as the fact that a record has not been tampered with or corrupted, either accident ally or maliciously. An authentic record is one that preserves the same identity it had when first generated, and can be presumed or proven to have maintained its integrity over time. The identity of a record is constituted of the whole of those characteristics that distinguish it from any other record, and is assessed on the basis of the formal elements on the face of the record, and/or its attributes, as expressed for example in a register entry or as metadata \cite[52]{duranti2009digital}.
The expectations of scholars with regard to the reliability of sources have evolved over the centuries, from the assumption that librarians and archivists would present researchers with evidence that could be relied upon to be verifiable, to more modern understandings that dispense with the ideal of the reliable source and consider all texts as potentially deceptive and richly ambiguous. Ideally, the methods of operation and processes developed by repositories over years of working with scholars and other patrons enable staff to provide researchers with documentation about the provenance and acquisition of the items in their care \cite[32]{kirschenbaum2010digital}.

\subsection{Accuracy}
Authenticity is the trustworthiness of a record as a record, and is defined as the fact that a record has not been tampered with or corrupted, either accident ally or maliciously. An authentic record is one that preserves the same identity it had when first generated, and can be presumed or proven to have maintained its integrity over time. The identity of a record is constituted of the whole of those characteristics that distinguish it from any other record, and is assessed on the basis of the formal elements on the face of the record, and/or its attributes, as expressed for example in a register entry or as metadata \cite[52]{duranti2009digital}.

\subsection{Integrity}
In the digital environment, there are no originals in the diplomatics sense, that is, there are no records which, in addition to being complete and capable of reaching the purposes for which they were generated (i.e., effective) are also the first instance of each item under consideration, because when we close a digital record for the first time we destroy the original and every time we open it we create a copy. However, we can state that each digital record, in the last version used by the creator in the usual and ordinary course of business, is a copy in the form of original and, in any version kept by the preserver, is an authentic copy of the record of the creator. They are both authoritative and authentic if their iden tity is intact and their integrity can be either presumed or proven. When extrating digital evidence, digital forensics must, first of all, avoid altering the data, and are guaranteed reliable in such sense by ensur ing that they are repeatable. Repeatability, which is one of the fundamental precepts of digital forensics practice, is supported by the accurate documentation of each and every action carried out on the evidence \cite[58]{duranti2009digital}. Such a chain of alteration is possible on the Ethereum network, since each update on an objects hash is natively documented on the blockchain.
Duplication integrity is ensured when given a data set, the process of creating a duplicate of the data does not modify the data (either intentionally or accidentally) and the duplicate is an exact bit copy of the original data set. It is possible to preserve data integrity over the duplicate, with respect to the original, by using a trusted third party. At the time the image is created, a copy of the hash can be given to a trusted third party to hold in escrow. Now changes to the duplicate can be detected even if the original is modified.
Digital forensics experts also link duplication integrity to time and have considered the use of time stamps for that purpose.63 A distinction between the integrity of a record as such and that of its duplicate may be useful to eliminate the conflict between the view of integrity held by diplomatists and that held by information technology experts, who tend to support the need for the extreme authentication provided by a digital signature. Indeed, one could further enrich the concept of integrity by also adopting the link between integrity and time proposed by digital forensics experts, and define record integrity differently in each phase of the record life cycle and/or custodial history \cite[60]{duranti2009digital}.
Both diplomatics and forensics were developed as practices for the purpose of investigating existing material evidence, assessing its status of transmission, its authenticity, and its ability to provide proof of facts at issue \cite[64]{duranti2009digital}.

\subsection{Authentication}
Authentication is defined as a declaration of authenticity made by a competent officer, and consists of a statement or an element, such as a seal, a stamp, or a symbol, added to the record after its completion. While authenticity is a quality of the record that accompanies it for as long as it exists as is, authentication only guarantees that a record is authentic at one specific moment in time, when the declaration is made or the authenticating element or entity is affixed. In the digital environment, extreme authentication is usually provided by a  igital signature. The digital signature has the function of a seal because it is attached to a complete record, allows verification of the origin and integrity of the record, and makes the record indisputable and incontestable by performing a non-repudiation function \cite[53]{duranti2009digital}.

\section{Data Integrity and Fixity}
\section{Fixity Checks}
\section{Security of Smart Contracts}
