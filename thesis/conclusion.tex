While the Ethereum blockchain has its limitation regarding the cost, the results of pooled testing showed in this thesis may include the blockchain in future implementations of fixity information storages. This research has shown that the operation cost can be reduced up to a third of cost, but also raises the question if that is enough to establish the Ethereum blockchain as a storage medium for fixity information due to its high cost. While the security and convenience of the Ethereum blockchain stands out, the cost of persisting fixity information for large ingest bulks in digital archives is still high, even after reducing the cost with pooled testing. Operators of digital archives may consider only persisting the fixity of hand-picked digital objects, where security and availability is needed the most, on the blockchain.

A digital archive could deploy the fixity storage on the Ethereum network by setting up an own Ethereum node or by utilizing the developer tools presented in this thesis. After deployment, the operator of the digital archive would be immediately able to persist fixity information on the blockchain after the installation of an Ethereum client of their choice, such as Pythons Web3.py. Due to its minimal requirements, the proposed solution is lightweight and does not require a larger infrastructure to be executed, since the infrastructure to store the fixity information is provided by the Ethereum blockchain.

The optimal pool size for an ingest bulk is dependent on the change rate of its object. For an ingest of N digital objects, the optimal pool size regarding the operation cost is N. With a pool of size N, you only have to write once on the blockchain and therefore minimize the cost. But, in order for a pooling strategy to be efficient, smaller pool sizes have to be chosen to minimize the amount of data-scrubbing operations. Therefore, pool sizes from 2 to 10 are favorable, depending on the change rate of the objects. A pool with size 10 is already at the limit of efficiency and can only be considered for non-volatile digital objects which are not expected to change and may only suffer from bit level errors.

With pooled testing, I was able to reduce the operation cost of preserving the fixity information of 1560 digital objects up to a third. The dataset used is the OpenPreserve format-corpus, an openly licensed dataset consisting of various file set and files from creation tools. The cost, utilizing an individual testing strategy was \$13218.81 whereas with a two-stage-hierarchical approach the cost was \$4406.27. In my thesis, I have also shown that the operation cost in USD can not be considered to be relevant, since it is heavily dependent on the price of gas and the Ether token, the native cryptocurrency of the Ethereum blockchain. In my thesis, I have chosen to present the operation cost in gas which is a constant value and does not change even when the \acrlong{eth} price rises or falls. The presented amount of gas can be converted into USD or any other currency. I have also shown, that the efficiency can be improved by a factor of 1.38 with the implementation of context-sensitive pooling. The context-sensitive approach is slightly more efficient than the two-stage-hierarchical pooling due to its consideration of inner group change rates, whereas two-stage-hierarchical pooling only considers a single change rate over the whole ingest bulk.
The third approach, utilizing split-off metadata has shown to have a worse the cost and efficiency compared to with non split-off metadata. Although, considering with split-off metadata you have the double amount of digital objects, the cost of this approach is still almost halved compared to individual testing.

Future work may include the migration of my proposed fixity information storage to another blockchain, a cheaper one. Although you have to consider that the price on the established Ethereum network has its security advantages, e.g. denial-of-service attacks only hurts the attacker because of the cost. A good consideration are so-called Layer 2 solutions, which is a term for Ethereum scaling solutions that handle transactions off Ethereum Layer 1 while still taking advantage of the security\footnote{\url{https://ethereum.org/en/layer-2/}}. Layer 2 solutions are considered faster and cheaper and therefore would be great candidate to enhance the proposed fixity information storage.\\
Future work may also include the estimation of change rates in ingest-bulks in order to compute the optimal pool size. In my work, I estimated the change rate of the format-corpus dataset by counting the involvement of certain file extensions in Git commits. My suggestion for the estimation is to monitor a digital archive for a certain amount of time and count the amount of times a certain object has changed. Another interesting method would be, to estimate the change rate of an object based on its file size. Assuming that a larger file has a higher chance to experience a bit level error during long term storage.

My contribution to archival science and preservation research was to gather the exact computational effort, in form of gas, needed to operate a decentralized fixity information storage on the Ethereum blockchain. Additionally, I have presented a method to decrease the amount of operations and therefore the amount of cost relevant transactions needed to operate such an application. With pooled testing, the operation cost of storing the fixity information for 1560 files in format-corpus dataset has been reduced to a third of the cost.
