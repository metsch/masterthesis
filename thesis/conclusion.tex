The optimal pool size can be calculated with two points in mind, when only regarding the cost you are able to increase the pool size until you have only one pool to ingest, whereas if you have the efficiency in mind you have to utilize smaller pool sizes utilize to minimize data-srcubbing actions on retrieval. I have presented two pooling strategies, two-hierarchical and context-sensitive pooling, which each of them is able to reduce the cost and increase efficiency. With pooled testing, I am guaranteed to decrease the cost of a decentralized fixity storage by at least 50\%, see Table \ref{tb:split-off} where each pooling strategy achieved a cost efficiency C(S) higher than 2, meaning the amount of writing transactions is at least halved. On the format-corpus dataset with prevalence rates estimated from git commits, the experiment with split-off metadata showed that, two-stage hierarchical pooled can decrease the cost even further whereas the context-sensitive approach shows weakness to the doubled amount of data with  high prevalence group such as the metadata, as shown in Table \ref{tb:context-sensitive}.
The research showed that varying price for tokens on the blockchain may be neglected from calculations, since they fluctuate too much. The focus was to find the absolute numbers of computational effort in form of gas, see Section \ref{sec:costs} for preserving fixity information, from which the value in EUR; \acrshort{eth} or other currencies may be derived at a later point. 
Future work may include the estimation of prevalence of corruption of ingest-bulks, my suggestion is to monitor a digital archive for a certain amount of time and count the amount of times a certain object has changed. 
My contribution to the research field is to gather the exact computational effort, in form of gas, needed to operate a decentralized fixity information storage on the Ethereum blockchain. Additionally, I have presented a strategy to decrease the amount of operations and therefore the amount of cost relevant transactions needed to operate such an application. With pooled testing, currently utilized in the COVID-19 pandemic, the amount of cost relevant \gls{transaction} has been reduced by at least 50\%.